\section{Sterowanie optymalne}

Jeżeli układ nieliniowy ma postać
\begin{equation}
\dot{x} = f(x) + g(x) \cdot u, \qquad x \in \mathbb{R}^n, u \in \mathbb{R}^n, t \in [0,T], f \in C^1
\end{equation}
z warunkiem początkowym
\begin{equation}
x(0) = x_0
\end{equation}
To przyjmując wskaźnik jakości w formie funkcjonału
\begin{equation} \label{wskaznik}
J(u, T) = q(T, x(T)) = T + \frac{1}{2} \rho |x(T) - x_{ref}|^2
\end{equation}
\textbf{sterowanie optymalne}, tj. takie, które najszybciej doprowadza układ do zadanego położenia ma postać sterowania bang-bang
\begin{equation}
u(t) = 
\begin{cases}
    u_{max}, g^T \psi > 0 \\
    u_{min}, g^T \psi < 0
\end{cases}
\end{equation}
Przyjęto że czasy przełączeń sterowania oznaczono przez $\tau_1, \tau_2, ..., \tau_n, T$.

\noindent
$\psi(t)$ to funkcja oznaczająca równanie sprzężone układu, takie że
\begin{equation}
\dot{\psi} = - \nabla_x H
\end{equation}
gdzie $H$ jest macierzą Hamiltona układu.

Sprawdzenie, czy sterowanie jest optymalne może być przeprowadzone przez wyrysowanie \textbf{funkcji przełączającej} i jej porównanie ze sterowaniem. Funkcja przełączająca ma postać
\begin{equation}
\Phi(t) = g^T(x(t)) \cdot \psi(t)
\end{equation}

\subsection{Analiza układu MagLev}

Hamiltonian układu \ref{maglev_rown} ma postać
\begin{equation}
H = \psi_1 x_2 + \psi_2 (bx_3^2 e^{ax_1}+g) + \psi_3 (cx_3 + u)
\end{equation}

Równania sprzężone układu magnetycznej lewitacji  \cite{Turnau}:
\begin{equation}
\begin{cases}
	\dot{\psi}_1 = - ab \cdot e^{ax_1} \cdot x_3^2 \psi_2 \\
	\dot{\psi}_2 = - \psi_1 \\
	\dot{\psi}_3 = -2b \cdot e^{ax_1} \cdot x_3 \psi_2 - c \psi_3
\end{cases}
\end{equation}

W celu przyspieszenia numerycznej optymalizacji sterowania z użyciem funkcji fmincon() z pakietu Optimization \textsc{MATLAB}a, należy obliczyć gradient wskaźnika jakości względem czasów przełączeń. Gradient wskaźnika jakości po czasach przełączeń $\tau_i$ dla układu MagLev jest dany wzorem
\begin{equation}
\nabla_{\tau_i} J = \Phi(t) (u^+ - u^-)
\end{equation}
Natomiast gradient wskaźnika jakości po czasie końcowym ma postać
\begin{equation}
\nabla_T J = 1 - H(\psi(T), x(T), u(T^-))
\end{equation}

%\begin{figure}[!htb]
%  \begin{center}
%    \includegraphics[width=14.5cm,trim=1.6cm 6.9cm 1.7cm 8.5cm,clip]
%    {img/exp_omega.pdf}
%  \end{center}
%  \caption{Eksperyment wyznaczenia charakterystyk prędkości śmigieł od napięcia na silnikach przy zablokowanych osiach}
%  \label{plot:exp1}
%\end{figure}





