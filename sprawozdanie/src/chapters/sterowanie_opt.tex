\section{Sterowanie optymalne}

Jeżeli układ nieliniowy ma postać
\begin{equation}
\dot{x} = f(x) + g(x) \cdot u
\end{equation}

Sterowanie optymalne

Funkcja przełączająca

Wskaźnik jakości

Gradient wskaźnika jakości
\cite{Turnau}

%\begin{figure}[!htb]
%  \begin{center}
%    \includegraphics[width=14.5cm,trim=1.6cm 6.9cm 1.7cm 8.5cm,clip]
%    {img/exp_omega.pdf}
%  \end{center}
%  \caption{Eksperyment wyznaczenia charakterystyk prędkości śmigieł od napięcia na silnikach przy zablokowanych osiach}
%  \label{plot:exp1}
%\end{figure}



\begin{table}[!htb]
  \centering
  \begin{tabular}{|c|l|l|}
  \hline
  Parametr/Funkcja & Wartość \\
  \hline
  $I_v$ & $0.0489$ \\
  \hline
  $l_m$ & $0.232$ \\
  \hline
  $F_m$ & -- \\
  \hline
  $f_v$ & $0.0142$ \\
  \hline
  $c_g$ & $0.252$ \\
  \hline
  $\phi_{v0}$ & $-0.536$ \\
  \hline
  $a_m$ & $0.0000329$ \\
  \hline
  $H^{-1}_m$ & -- \\
  \hline
  $c_{mm}$ & $326$ \\
  \hline
  \end{tabular}
  \caption{Parametry modelu}
  \label{tab:idf}
\end{table}



