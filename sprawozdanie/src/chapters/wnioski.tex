\section{Wnioski}

Sterowanie docelowe metodą QTO-RHC pozwala na predykcję funkcji sterującej dla obiektów nieliniowych. 
Często znalezienie sterowania optymalnego jest trudnym zadaniem, które wymaga dużych nakładów 
obliczeniowych i czasowych. Naszym zadaniem optymalizacji była stabilizacja położenia i prędkości metalowej
kulki lewitującej w polu magnetycznym. W trakcie laboratoriów wykorzystywano algorytm Rungego-Kutty, za 
pomocą którego rozwiązywano równania różniczkowe "w przód" oraz "wstecz". Została napisana funkcja do
obliczania wskaźnika jakości wraz z gradientami, która następnie została poddana minimalizacji przy pomocy
funckji fmincon programu Matlab. Jako jedną z opcji funkcji fmincon ustawiono wykorzystanie obliczanego 
wewnątrz funkcji celu gradientu wskaźnika jakości. Przyspieszyło to znalezienie sterowania odpowiadającego
założeniom projektowym. Rozwiązanie problemu optymalizacji było możliwe dzięki wcześniejszej znajomości
modelu matematycznego rozważanego obiektu wraz z równaniami stanu i sprzężonymi. 
