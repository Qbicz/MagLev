\section{Wstęp}

Sterowanie optymalne - znaczenie tego wyrażenia nie jest od razu jasne. Zależy bowiem przede wszystkim od tego, jakie kryterium oceny przyjmiemy. W klasycznym rozumieniu regulatorem optymalnym jest regulator liniowo-kwadratowy LQ, który minimalizuje wskaźnik biorący pod uwagę uchyb oraz koszt energetyczny sterowania. Co jednak, jeśli chcemy jak najszybciej osiągnąć efekt i przeprowadzić układ do żądanego stanu, nie zważając na jego zachowanie \quotedblbase w drodze\textquotedblright? Rozum podpowiada, że należy układ pobudzić największym dostępnym sterowaniem, a zanim dotrzemy do zadanego punktu, maksymalnym przeciwnym sterowaniem go wyhamować. Rzeczywiście, w pracy \cite{athans-falb} dowiedziono że odpowiednio obliczone sterowanie \textit{bang-bang} jest rozwiązaniem problemu sterowania docelowego.

Podczas przedmiotu \textit{Optymalizacja w systemach sterowania} poznaliśmy metodykę wyznaczania sterowania predykcyjnego. W oparciu o pracę \cite{Bania} przygotowaliśmy algorytm optymalizacji sterowania docelowego układu lewitacji magnetycznej.

\vspace{20pt}
Cele projektu:
\begin{enumerate}
  \item Zapoznanie się z algorytmem QTO-RHC proponowanym w \cite{Bania}
  \item Realizacja rozwiązywania równań różniczkowych układu MagLev w przód
  \item Rozwiązywanie równań sprzężonych w tył
  \item Przyjęcie wskaźnika jakości regulacji
  \item Wyznaczenie gradientu wskaźnika jakości po czasach przełączeń i czasie końcowym
  \item Optymalizacja numeryczna sterowania bang-bang względem założonego wskaźnika
\end{enumerate}
