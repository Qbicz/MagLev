\section{Wnioski}

Podczas zajęć przygotowaliśmy model matematyczny stanowiska lewitacji magnetycznej, wykonaliśmy identyfikację parametrów modelu oraz zaprojektowaliśmy ciągły regulator LQ z niezerową wartością zadaną dla zlinearyzowanego modelu.

Otrzymany model nie jest dobrze zbieżny z rzeczywistym obiektem. Spowodowało to oczywiste trudności przy linearyzacji i projektowaniu regulatora liniowo-kwadratowego. Regulator stabilizujący model okazał się nie zapewniać stabilności obiektu w laboratorium.

Do sukcesów możemy zaliczyć sprawną współpracę dzięki systemowi kontroli wersji Git. Jego użycie praktycznie wyeliminowało obawy o aktualność używanych plików, danych i modeli w obrębie wszystkich używanych komputerów.

Jako przyczyny niewystarczającego dostrojenia modelu możemy podać małą ilość czasu w laboratorium - pierwsze kilka spotkań poświęciliśmy na obsługę niedawno zmodernizowanego stanowiska, później z kolei współdzieliliśmy stanowisko ze studentami, którzy również chcieli przetestować swoje metody regulacji na rzeczywistym obiekcie.
