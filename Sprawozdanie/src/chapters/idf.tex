\section{Identyfikacja}


\subsection{Identyfikacja charakterystyki czujnika położenia}

Pomiar położenia sfery w układzie magnetycznej lewitacji jest dokonywany optycznie. Z jednej strony znajduje się źródło światła, a po przeciwnej stronie fotodioda z przetwornikiem A/C, która podaje pewne napięcie $u_x$. Podczas identyfikacji poszukujemy zależności tego napięcia od położenia sfery:

\begin{equation} \label{eq:gx}
u_x = g(x_1)
\end{equation}

Poszukujemy charakterystyki statycznej $g(x_1)$, którą otrzymamy przykręcając sferę do śruby i podnosząc ją co ustalony skok 0,7 mm. Za każdym razem dokonujemy pomiaru napięcia podanego przez detektor światła.

Do pracy z modelem potrzebna jest znajomość położenia sfery, dlatego na rysunku \ref{img:gx} charakterystyka odwrotną do zależności \ref{eq:gx}.

[trzeba przeskalować napięcie jeszcze, można zrobić wykres od -ux]

\begin{figure}[!htb]
\centering
\includegraphics[scale=1]{img/czujnik_polozenia.png}
\caption{Charakterystyka statyczna optycznego czujnika położenia}
\label{img:gx}
\end{figure}


W pracy \cite{Bania1999} autor dokonał aproksymacji otrzymanej charakterystyki odwrotnej sumą funkcji wykładniczych metodą prób i błędów. Nie będziemy dokonywać takiej aproksymacji, ponieważ podczas pracy z modelem w laboratorium użyjemy bloku \textit{LUT z interpolacją} oferowanego przez Simulink.



\subsection{Identyfikacja parametrów cewki $k, T, u_c$ }

Aby wiedzieć, jak zmienia się prąd cewki w zależności od użytego sterowania, czyli przyłożonego napięcia $u$, należy wyznaczyć parametry $k, T$ oraz $u_c$.

\subsubsection{Pomiary w stanie ustalonym cewki}

Zależność prądu od napięcia jest liniowa
\begin{equation}
i = k(u + u_c)
\end{equation}

Parametry $k$ i $u_c$ (wzmocnienie oraz stałe napięcie na cewce) wyznaczymy mierząc prąd w stanie ustalonym dla różnych wartości napięcia sterującego.

- prąd w zależności od napięcia sterującego pomierzony, jakie jest przeskalowanie? Czy lepiej użyć oscyloskopu / czy jest rezystor?

[można też mierzyć spadek napięcia na rezystorze pomiarowym bo są duże błędy prądu]

\begin{figure}[!htb]
\centering
\includegraphics[scale=0.85]{img/identyfikacja_cewki.png}
\caption{Identyfikacja parametrów statycznych cewki}
\label{rys:cewka_k_uc}
\end{figure}

\subsubsection{Pomiary stanów przejściowych cewki}

Stałą czasową $T$ można wyznaczyć obserwując odpowiedź skokową prądu.


...

...





Korzystając z metody najmniejszych kwadratów wyznaczono parametry, których wartości umieszczono w tabeli.

\begin{tabular}{c r @{,} l}
Wyrażenie &
\multicolumn{2}{c}{Wartość}\\ \hline
$k$ & ... \\
$T$ & ...&.. \\
$u_c$ & ..&. \\
\end{tabular}



\subsection{Identyfikacja indukcyjności cewki $L(x_1)$}

W celu identyfikacji zależności indukcyjności cewki od położenia w układzie otwartym należy wykonać serię pomiarów napięcia i prądu dla różnych położeń sfery. Zmierzona rezystancja cewki wynosi $R = ..$. Indukcyjność obliczymy ze wzoru
\begin{equation}
L = \dfrac{1}{\omega}\sqrt{\dfrac{U^2}{I^2} - R^2}
\end{equation}

gdzie
$\omega$ - częstość napięcia zasilającego ($\omega$ = 314 rad/s)
$U$ - napięcie skuteczne na cewce [V]
$I$ - prąd płynący przez cewkę [A]
$R$ - rezystancja cewki

[wykres doświadczenia]

Znaleziona funkcja ma postać
$L(x_1) = ...$


%\begin{figure}[!htb]
%\centering
%\includegraphics[scale=0.85]{img/gasnace.png}
%\caption{Identyfikacja oporów tarcia belki w płaszczyźnie pionowej}
%\label{rys:gasnace}
%\end{figure}


