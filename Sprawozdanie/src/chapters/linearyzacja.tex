\section{Regulator liniowo-kwadratowy}

Po weryfikacji modelu, równania zostały zlinearyzowane. Przyjęto kilka punktów równowagi: 12mm, 14mm, 16mm i 18mm, aby móc przełączać otrzymany później regulator podczas pracy układu i stabilizować go w różnych punktach pracy.

\subsection{Linearyzacja}

Linearyzacji modelu nieliniowego dokonuje się w otoczeniu punktu równowagi, zastępując nieliniowe równania stanu
\begin{equation}
\dot{x} = f(x)
\end{equation}
liniowymi równaniami, które można przedstawić w postaci macierzowej
\begin{equation}
\dot{x} = Ax + Bu
\end{equation}

Aby otrzymać macierz stanu A, należy wyznaczyć macierz Jacobiego pierwszych pochodnych
\begin{equation}
J = \dfrac{\partial f}{\partial x}(x)
\end{equation}
a następnie obliczyć jej wartości dla poszczególnych punktów stacjonarnych $x*$
\begin{equation}
A = J(x*) = \dfrac{\partial f}{\partial x}(x*)
\end{equation}

Dla równań magnetycznej lewitacji (\ref{modelMagLev}) zlinearyzowana macierz ma postać

\[
\begin{array}{lc}
A = &
\begin{bmatrix} 0 & 1 & 0 \\ \dfrac{2 \cdot 10^{-3} agx_3^2}{(ax_1+b)^3} & 0 & \dfrac{-2 \cdot 10^{-3} gx_3}{(ax_1+b)^2} \\ 0 & 0 & -\dfrac{1}{T}
\end{bmatrix}
\end{array}
\]

W punkcie równowagi $ x_{0_{14}} = 0,014 m $

\subsection{Synteza regulatora LQ}


